\hypertarget{yttrius-documentations}{%
\section{Yttrius Documentations}\label{yttrius-documentations}}

\begin{center}\rule{0.5\linewidth}{0.5pt}\end{center}

\textbf{Support Server:} \href{https://discord.gg/PbJQRT9zQ8}{Click
Here} \textbf{Available Download Types:} \texttt{TXT} \texttt{MD}
\texttt{MDX*} \texttt{HTML} \texttt{MHTML} \texttt{RTXT*}

** - being worked on* \_\_\_ \#\#\# Section 0 - Introduction This is a
curated list of information regarding Yttrius, the bot, and the
framework(s).

\textbf{All Available Sections} \textgreater{} Section 0 - Introduction

\begin{quote}
Section 1.0 - Documentation Informations
\end{quote}

\begin{quote}
Section 1.1 - Internal Structuring
\end{quote}

\begin{quote}
Section 1.2 - Yttrius Introduction
\end{quote}

\begin{quote}
Section 1.2.1 - Yttrius Information
\end{quote}

\begin{quote}
Section 1.3 - Yttrius Discord Bot Usages
\end{quote}

\begin{quote}
Section 1.4 - Commands Categories Explanation \_\_\_ \#\#\# Section 1.0
- Documentational Informations
\end{quote}

\textbf{Documentation Version:} \texttt{1.0\ (LATEST-UNFINISHED)}

\textbf{Documentation Type:} \texttt{MARKDOWN}

\textbf{Bot Version:} \texttt{1.7\ (STABLE-BRANCH@62bc834)}

\textbf{Assets Repo Version:} \texttt{3\ (UNCHECKED)}

\textbf{License:} \texttt{Closed\ Source}

\textbf{Package Licensing:}
\texttt{ISC\ \textbar{}\ BSD-3\ \textbar{}\ MIT}

\textbf{Package Manager:} \texttt{snapd} \texttt{yum} \texttt{vcpkg}
\texttt{conan} \texttt{cget}

\textbf{Documentation Last Revision:} \texttt{September-2-2021} \_\_\_
\#\#\# Section 1.1 - Internal Structuring

\textbf{Language(s):} \texttt{en-US}

\textbf{Programming Language(s):} \texttt{C++} \texttt{JavaScript}
\texttt{Elixir}

\textbf{Archival Language(s):} \texttt{JSON} \texttt{YAML} \texttt{XML}
\texttt{TXT} \texttt{RTXT}

\textbf{Frameworks:} \texttt{sleepy-discord} \texttt{NodeJS}
\texttt{Sugar}

\textbf{Operators / Compilation Handlers:} \texttt{NPM} \texttt{YARN}
\texttt{CMAKE}

\textbf{Host Platform:} \texttt{Ubuntu-16.04LTS} \_\_\_

\hypertarget{section-1.2---yttrius-introduction}{%
\subsubsection{Section 1.2 - Yttrius
Introduction}\label{section-1.2---yttrius-introduction}}

Yttrius is written mainly in C++ using this library
\href{https://github.com/yourWaifu/sleepy-discord}{here}. However, it
also incorporates compatibility with JavaScript \& TypeScript in order
to allow more efficient packages from YARN \& NPM to be installed.

\begin{center}\rule{0.5\linewidth}{0.5pt}\end{center}

\hypertarget{section-1.2.1---yttrius-information}{%
\subsubsection{Section 1.2.1 - Yttrius
Information}\label{section-1.2.1---yttrius-information}}

Yttrius itself is not the bot, no no. It is a framework I can use to
make Discord Bots and other online automated services. However, in this
case it is a Discord Bot.

\textbf{Bot Information}

User Friendliness Level (1 least 5 most): \texttt{3}

Response Time (1 Slowest, 10 Fastest): \texttt{7}

APIs Overtime (1 most, 5 least): \texttt{1}

Error Handling Depth (1 least 5 most): \texttt{4}

Self User Debugging? \texttt{y}

Unhandled Errors Expected (1 always, 3 never): \texttt{3}

Profanity Filtering Level (1 none 5 strictest): \texttt{5}

Auto-Logging? \texttt{y}

Data Collection (1 security, 5 all): \texttt{2} \_\_\_

\hypertarget{section-1.3---yttrius-discord-bot-usages}{%
\subsubsection{Section 1.3 - Yttrius Discord Bot
Usages}\label{section-1.3---yttrius-discord-bot-usages}}

All commands if input is required will specify when the command is read
without necessary inputs. If no input is detected, a help menu of the
command will be displayed along with the appropriate format for the
command.

This bot requires the strictest of command formatting. If a necessary
input is required and is not received within the user input, then the
whole operation will cease.

\textbf{Prefix Retrieval:} \texttt{@Yttrius}

\textbf{Prefix:} \texttt{\$}

\textbf{Help Menu:} \texttt{\$help}

\textbf{Command Help Seek} \texttt{\$cmd} \_\_\_

\hypertarget{section-1.4---commands-categories-explanation}{%
\subsubsection{Section 1.4 - Commands Categories
Explanation}\label{section-1.4---commands-categories-explanation}}

This section will explain the different categories of topics of commands
within the bot.

\textbf{Main}

Main Commands List

\texttt{bot}
